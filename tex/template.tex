\documentclass{jsreport}
\usepackage{../sty/thesis}
% \usepackage[usefigtable]{../sty/thesis}	% 図表をFig.とTableとする場合はこちらを使う
% \usepackage[postshiki]{../sty/thesis} % \cref参照時に「式」を後ろに持ってくる場合はこちらを使う
% \usepackage[usefigtable,postshiki]{../sty/thesis}	% 上記両方を指定する場合はこちらを使う


\begin{document}

\fyear{xx}   % 年度
\papername{〇〇論文}	% 論文名(卒業論文・修士論文など)
\title{タイトル}  % タイトル
\id{xxxxxxx}    % 学籍番号
\author{〇〇 〇〇}  % 著者
\university{〇〇大学}	% 大学名
\cource{〇〇研究科}	% 研究科名
\major{〇〇専攻}	% 専攻名
\supervisor{〇〇 〇〇}{教授}	% 指導教員を人数分設定(引数1に名前,引数2に職階を指定)
\supervisor{〇〇 〇〇}{准教授}
\supervisor{〇〇 〇〇}{准教授}
\postdate{平成xx年xx月xx日}  % 提出年月日


\maketitle

\tableofcontents
\clearpage


\chapter{章タイトル}

\section{節タイトル}

\subsection{小節タイトル}


\chapter*{謝辞}
\addcontentsline{toc}{chapter}{謝辞}


\begin{thebibliography}{99}

\end{thebibliography}

\end{document}
